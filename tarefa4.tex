% Options for packages loaded elsewhere
\PassOptionsToPackage{unicode}{hyperref}
\PassOptionsToPackage{hyphens}{url}
\PassOptionsToPackage{dvipsnames,svgnames,x11names}{xcolor}
%
\documentclass[
  12pt]{article}

\usepackage{amsmath,amssymb}
\usepackage{iftex}
\ifPDFTeX
  \usepackage[T1]{fontenc}
  \usepackage[utf8]{inputenc}
  \usepackage{textcomp} % provide euro and other symbols
\else % if luatex or xetex
  \usepackage{unicode-math}
  \defaultfontfeatures{Scale=MatchLowercase}
  \defaultfontfeatures[\rmfamily]{Ligatures=TeX,Scale=1}
\fi
\usepackage{lmodern}
\ifPDFTeX\else  
    % xetex/luatex font selection
\fi
% Use upquote if available, for straight quotes in verbatim environments
\IfFileExists{upquote.sty}{\usepackage{upquote}}{}
\IfFileExists{microtype.sty}{% use microtype if available
  \usepackage[]{microtype}
  \UseMicrotypeSet[protrusion]{basicmath} % disable protrusion for tt fonts
}{}
\makeatletter
\@ifundefined{KOMAClassName}{% if non-KOMA class
  \IfFileExists{parskip.sty}{%
    \usepackage{parskip}
  }{% else
    \setlength{\parindent}{0pt}
    \setlength{\parskip}{6pt plus 2pt minus 1pt}}
}{% if KOMA class
  \KOMAoptions{parskip=half}}
\makeatother
\usepackage{xcolor}
\setlength{\emergencystretch}{3em} % prevent overfull lines
\setcounter{secnumdepth}{5}
% Make \paragraph and \subparagraph free-standing
\ifx\paragraph\undefined\else
  \let\oldparagraph\paragraph
  \renewcommand{\paragraph}[1]{\oldparagraph{#1}\mbox{}}
\fi
\ifx\subparagraph\undefined\else
  \let\oldsubparagraph\subparagraph
  \renewcommand{\subparagraph}[1]{\oldsubparagraph{#1}\mbox{}}
\fi


\providecommand{\tightlist}{%
  \setlength{\itemsep}{0pt}\setlength{\parskip}{0pt}}\usepackage{longtable,booktabs,array}
\usepackage{calc} % for calculating minipage widths
% Correct order of tables after \paragraph or \subparagraph
\usepackage{etoolbox}
\makeatletter
\patchcmd\longtable{\par}{\if@noskipsec\mbox{}\fi\par}{}{}
\makeatother
% Allow footnotes in longtable head/foot
\IfFileExists{footnotehyper.sty}{\usepackage{footnotehyper}}{\usepackage{footnote}}
\makesavenoteenv{longtable}
\usepackage{graphicx}
\makeatletter
\def\maxwidth{\ifdim\Gin@nat@width>\linewidth\linewidth\else\Gin@nat@width\fi}
\def\maxheight{\ifdim\Gin@nat@height>\textheight\textheight\else\Gin@nat@height\fi}
\makeatother
% Scale images if necessary, so that they will not overflow the page
% margins by default, and it is still possible to overwrite the defaults
% using explicit options in \includegraphics[width, height, ...]{}
\setkeys{Gin}{width=\maxwidth,height=\maxheight,keepaspectratio}
% Set default figure placement to htbp
\makeatletter
\def\fps@figure{htbp}
\makeatother

\addtolength{\oddsidemargin}{-.5in}%
\addtolength{\evensidemargin}{-1in}%
\addtolength{\textwidth}{1in}%
\addtolength{\textheight}{1.7in}%
\addtolength{\topmargin}{-1in}%
\makeatletter
\makeatother
\makeatletter
\makeatother
\makeatletter
\@ifpackageloaded{caption}{}{\usepackage{caption}}
\AtBeginDocument{%
\ifdefined\contentsname
  \renewcommand*\contentsname{Table of contents}
\else
  \newcommand\contentsname{Table of contents}
\fi
\ifdefined\listfigurename
  \renewcommand*\listfigurename{List of Figures}
\else
  \newcommand\listfigurename{List of Figures}
\fi
\ifdefined\listtablename
  \renewcommand*\listtablename{List of Tables}
\else
  \newcommand\listtablename{List of Tables}
\fi
\ifdefined\figurename
  \renewcommand*\figurename{Figure}
\else
  \newcommand\figurename{Figure}
\fi
\ifdefined\tablename
  \renewcommand*\tablename{Table}
\else
  \newcommand\tablename{Table}
\fi
}
\@ifpackageloaded{float}{}{\usepackage{float}}
\floatstyle{ruled}
\@ifundefined{c@chapter}{\newfloat{codelisting}{h}{lop}}{\newfloat{codelisting}{h}{lop}[chapter]}
\floatname{codelisting}{Listing}
\newcommand*\listoflistings{\listof{codelisting}{List of Listings}}
\makeatother
\makeatletter
\@ifpackageloaded{caption}{}{\usepackage{caption}}
\@ifpackageloaded{subcaption}{}{\usepackage{subcaption}}
\makeatother
\makeatletter
\@ifpackageloaded{tcolorbox}{}{\usepackage[skins,breakable]{tcolorbox}}
\makeatother
\makeatletter
\@ifundefined{shadecolor}{\definecolor{shadecolor}{rgb}{.97, .97, .97}}
\makeatother
\makeatletter
\makeatother
\makeatletter
\makeatother
\ifLuaTeX
  \usepackage{selnolig}  % disable illegal ligatures
\fi
\usepackage[]{natbib}
\bibliographystyle{agsm}
\IfFileExists{bookmark.sty}{\usepackage{bookmark}}{\usepackage{hyperref}}
\IfFileExists{xurl.sty}{\usepackage{xurl}}{} % add URL line breaks if available
\urlstyle{same} % disable monospaced font for URLs
\hypersetup{
  pdftitle={AlgaeMAp: Algae Bloom Monitoring Application for Inland Waters in Latin America},
  pdfauthor={Felipe de Lucia Lobo; Gustavo Willy Nagel},
  pdfkeywords={Google Earth Engine, Sentinel-2, water
quality, chlorophyll-a, Trophic State Index, Earth Engine App},
  colorlinks=true,
  linkcolor={blue},
  filecolor={Maroon},
  citecolor={Blue},
  urlcolor={Blue},
  pdfcreator={LaTeX via pandoc}}


\begin{document}


\def\spacingset#1{\renewcommand{\baselinestretch}%
{#1}\small\normalsize} \spacingset{1}


%%%%%%%%%%%%%%%%%%%%%%%%%%%%%%%%%%%%%%%%%%%%%%%%%%%%%%%%%%%%%%%%%%%%%%%%%%%%%%

\date{July 10, 2023}
\title{\bf AlgaeMAp: Algae Bloom Monitoring Application for Inland
Waters in Latin America}
\author{
Felipe de Lucia Lobo\thanks{CNPq, Capes, INPE}\\
Department of Technological Development (CDTec), Federal University of
Pelotas\\
and\\Gustavo Willy Nagel\\
Remote Sensing Program, National Institute for Space Research (INPE)\\
}
\maketitle

\bigskip
\bigskip
\begin{abstract}
Due to increasing algae bloom occurrence and water degradation on a
global scale, there is a demand for water quality monitoring systems
based on remote sensing imagery. This paper describes the scientific,
theoretical, and methodological background for creating a
cloud-computing interface on Google Earth Engine (GEE) which allows
end-users to access algae bloom related products with high spatial (30
m) and temporal (\textasciitilde5 day) resolution. The proposed
methodology uses Sentinel-2 images corrected for atmospheric and
sun-glint effects to generate an image collection of the Normalized
Difference Chlorophyll-a Index (NDCI) for the entire time-series. NDCI
is used to estimate both Chl-a concentration, based on a non-linear
fitting model, and Trophic State Index (TSI), based on a tree-decision
model classification into five classes. Once the Chl-a and TSI
algorithms had been calibrated and validated they were implemented in
GEE as an Earth Engine App, entitled Algae Bloom Monitoring Application
(AlgaeMAp). AlgaeMAp is the first online platform built within the GEE
platform that offers high spatial resolution of water quality
parameters. The App benefits from the huge processing capability of GEE
that allows any user with internet access to easily extract detailed
spatial (30 m) and long temporal Chl-a and TSI information.
\end{abstract}

\noindent%
{\it Keywords:} Google Earth Engine, Sentinel-2, water
quality, chlorophyll-a, Trophic State Index, Earth Engine App
\vfill

\newpage
\spacingset{1.9} % DON'T change the spacing!
\ifdefined\Shaded\renewenvironment{Shaded}{\begin{tcolorbox}[frame hidden, sharp corners, borderline west={3pt}{0pt}{shadecolor}, breakable, interior hidden, boxrule=0pt, enhanced]}{\end{tcolorbox}}\fi

\hypertarget{sec-intro}{%
\section{Introduction}\label{sec-intro}}

Body of paper. Margins in this document are roughly 0.75 inches all
around, letter size paper.

\begin{figure}

{\centering \includegraphics[width=3in,height=\textheight]{fig1.pdf}

}

\caption{\label{fig-first}Consistency comparison in fitting surrogate
model in the tidal power example.}

\end{figure}

\hypertarget{tbl-one}{}
\begin{longtable}[]{@{}lllll@{}}
\caption{\label{tbl-one}D-optimality values for design X under five
different scenarios.}\tabularnewline
\toprule\noalign{}
one & two & three & four & five \\
\midrule\noalign{}
\endfirsthead
\toprule\noalign{}
one & two & three & four & five \\
\midrule\noalign{}
\endhead
\bottomrule\noalign{}
\endlastfoot
1.23 & 3.45 & 5.00 & 1.21 & 3.41 \\
1.23 & 3.45 & 5.00 & 1.21 & 3.42 \\
1.23 & 3.45 & 5.00 & 1.21 & 3.43 \\
\end{longtable}

\begin{itemize}
\tightlist
\item
  Note that figures and tables (such as Figure~\ref{fig-first} and
  Table~\ref{tbl-one}) should appear in the paper, not at the end or in
  separate files.
\item
  In document front matter, you may set the key \texttt{blinded} under a
  \texttt{journal} key to hide the authors and acknowledgements,
  producing the required anonymized version.
\item
  Remember that in the anonymized version, you should not identify
  authors indirectly in the text. That is, don't say ``In Smith et.
  al.~(2009) we showed that \ldots{}''. Instead, say ``Smith et.
  al.~(2009) showed that \ldots{}''.
\item
  These points are only intended to remind you of some requirements.
  Please refer to the instructions for authors at
  \url{http://amstat.tandfonline.com/action/authorSubmission?journalCode=uasa20\&page=instructions\#.VFkk7fnF_0c}
\item
  For more about ASA~style, please see
  \url{https://files.taylorandfrancis.com/asa-style-guide.pdf}.
\item
  If you have supplementary material (e.g., software, data, technical
  proofs), identify them in the section below. In early stages of the
  submission process, you may be unsure what to include as supplementary
  material. Don't worry---this is something that can be worked out at
  later stages.
\end{itemize}

\hypertarget{sec-meth}{%
\section{Methods}\label{sec-meth}}

Don't take any of these section titles seriously. They're just for
illustration.

\hypertarget{sec-verify}{%
\section{Verifications}\label{sec-verify}}

This section will be just long enough to illustrate what a full page of
text looks like, for margins and spacing.

\addtolength{\textheight}{.5in}%

\citet{gelm:veht:2021} offer some guidance about key ideas about
statistical ideas. On an unrelated note, spreadsheets are important to
use correctly \citep{brom:woo:2018}. Log-linear models are an attractive
way to model categorical data \citep{bish:fien:1975}.

The quick brown fox jumped over the lazy dog. The quick brown fox jumped
over the lazy dog. The quick brown fox jumped over the lazy dog. The
quick brown fox jumped over the lazy dog. \textbf{With this spacing we
have 25 lines per page.} The quick brown fox jumped over the lazy dog.
The quick brown fox jumped over the lazy dog. The quick brown fox jumped
over the lazy dog. The quick brown fox jumped over the lazy dog. The
quick brown fox jumped over the lazy dog.

The quick brown fox jumped over the lazy dog. The quick brown fox jumped
over the lazy dog. The quick brown fox jumped over the lazy dog. The
quick brown fox jumped over the lazy dog. The quick brown fox jumped
over the lazy dog. The quick brown fox jumped over the lazy dog. The
quick brown fox jumped over the lazy dog. The quick brown fox jumped
over the lazy dog. The quick brown fox jumped over the lazy dog. The
quick brown fox jumped over the lazy dog.

The quick brown fox jumped over the lazy dog. The quick brown fox jumped
over the lazy dog. The quick brown fox jumped over the lazy dog. The
quick brown fox jumped over the lazy dog. The quick brown fox jumped
over the lazy dog. The quick brown fox jumped over the lazy dog. The
quick brown fox jumped over the lazy dog. The quick brown fox jumped
over the lazy dog. The quick brown fox jumped over the lazy dog. The
quick brown fox jumped over the lazy dog.

The quick brown fox jumped over the lazy dog. The quick brown fox jumped
over the lazy dog. The quick brown fox jumped over the lazy dog. The
quick brown fox jumped over the lazy dog. The quick brown fox jumped
over the lazy dog. The quick brown fox jumped over the lazy dog. The
quick brown fox jumped over the lazy dog. The quick brown fox jumped
over the lazy dog. The quick brown fox jumped over the lazy dog. The
quick brown fox jumped over the lazy dog.

\addtolength{\textheight}{-.5in}%

\addtolength{\textheight}{.2in}%

The quick brown fox jumped over the lazy dog. The quick brown fox jumped
over the lazy dog. The quick brown fox jumped over the lazy dog. The
quick brown fox jumped over the lazy dog. The quick brown fox jumped
over the lazy dog. The quick brown fox jumped over the lazy dog. The
quick brown fox jumped over the lazy dog. The quick brown fox jumped
over the lazy dog. The quick brown fox jumped over the lazy dog. The
quick brown fox jumped over the lazy dog.

The quick brown fox jumped over the lazy dog. The quick brown fox jumped
over the lazy dog. The quick brown fox jumped over the lazy dog. The
quick brown fox jumped over the lazy dog. The quick brown fox jumped
over the lazy dog. The quick brown fox jumped over the lazy dog. The
quick brown fox jumped over the lazy dog. The quick brown fox jumped
over the lazy dog. The quick brown fox jumped over the lazy dog. The
quick brown fox jumped over the lazy dog.

The quick brown fox jumped over the lazy dog. The quick brown fox jumped
over the lazy dog. The quick brown fox jumped over the lazy dog. The
quick brown fox jumped over the lazy dog. The quick brown fox jumped
over the lazy dog. The quick brown fox jumped over the lazy dog. The
quick brown fox jumped over the lazy dog. The quick brown fox jumped
over the lazy dog. The quick brown fox jumped over the lazy dog. The
quick brown fox jumped over the lazy dog.

The quick brown fox jumped over the lazy dog. The quick brown fox jumped
over the lazy dog. The quick brown fox jumped over the lazy dog. The
quick brown fox jumped over the lazy dog. The quick brown fox jumped
over the lazy dog. The quick brown fox jumped over the lazy dog. The
quick brown fox jumped over the lazy dog. The quick brown fox jumped
over the lazy dog. The quick brown fox jumped over the lazy dog. The
quick brown fox jumped over the lazy dog.

The quick brown fox jumped over the lazy dog. The quick brown fox jumped
over the lazy dog. The quick brown fox jumped over the lazy dog. The
quick brown fox jumped over the lazy dog. The quick brown fox jumped
over the lazy dog. The quick brown fox jumped over the lazy dog. The
quick brown fox jumped over the lazy dog. The quick brown fox jumped
over the lazy dog. The quick brown fox jumped over the lazy dog. The
quick brown fox jumped over the lazy dog.

The quick brown fox jumped over the lazy dog. The quick brown fox jumped
over the lazy dog. The quick brown fox jumped over the lazy dog. The
quick brown fox jumped over the lazy dog. The quick brown fox jumped
over the lazy dog. The quick brown fox jumped over the lazy dog. The
quick brown fox jumped over the lazy dog. The quick brown fox jumped
over the lazy dog. The quick brown fox jumped over the lazy dog. The
quick brown fox jumped over the lazy dog.

The quick brown fox jumped over the lazy dog. The quick brown fox jumped
over the lazy dog. The quick brown fox jumped over the lazy dog. The
quick brown fox jumped over the lazy dog.

\addtolength{\textheight}{-.2in}%

\hypertarget{sec-conc}{%
\section{Conclusion}\label{sec-conc}}

\hypertarget{disclosure-statement}{%
\section{Disclosure statement}\label{disclosure-statement}}

The authors have the following conflicts of interest to declare (or
replace with a statement that no conflicts of interest exist).

\hypertarget{data-availability-statement}{%
\section{Data Availability
Statement}\label{data-availability-statement}}

Deidentified data have been made available at the following URL: XX.

\hypertarget{supplementary-material}{}
\bigskip

\begin{center}

{\large\bf SUPPLEMENTARY MATERIAL}

\end{center}

\begin{description}
\item[Title:]
Brief description. (file type)
\item[R-package for MYNEW routine:]
R-package MYNEW containing code to perform the diagnostic methods
described in the article. The package also contains all datasets used as
examples in the article. (GNU zipped tar file)
\item[HIV data set:]
Data set used in the illustration of MYNEW method in
Section~\ref{sec-verify} (.txt file).
\end{description}

\hypertarget{bibtex}{%
\section{BibTeX}\label{bibtex}}

We encourage you to use BibTeX. If you have, please feel free to use the
package natbib with any bibliography style you're comfortable with. The
.bst file agsm has been included here for your convenience.


  \bibliography{bibliography.bib}


\end{document}
