% Options for packages loaded elsewhere
\PassOptionsToPackage{unicode}{hyperref}
\PassOptionsToPackage{hyphens}{url}
\PassOptionsToPackage{dvipsnames,svgnames,x11names}{xcolor}
%
\documentclass[
  super,
  preprint,
  3p]{elsarticle}

\usepackage{amsmath,amssymb}
\usepackage{iftex}
\ifPDFTeX
  \usepackage[T1]{fontenc}
  \usepackage[utf8]{inputenc}
  \usepackage{textcomp} % provide euro and other symbols
\else % if luatex or xetex
  \usepackage{unicode-math}
  \defaultfontfeatures{Scale=MatchLowercase}
  \defaultfontfeatures[\rmfamily]{Ligatures=TeX,Scale=1}
\fi
\usepackage{lmodern}
\ifPDFTeX\else  
    % xetex/luatex font selection
\fi
% Use upquote if available, for straight quotes in verbatim environments
\IfFileExists{upquote.sty}{\usepackage{upquote}}{}
\IfFileExists{microtype.sty}{% use microtype if available
  \usepackage[]{microtype}
  \UseMicrotypeSet[protrusion]{basicmath} % disable protrusion for tt fonts
}{}
\makeatletter
\@ifundefined{KOMAClassName}{% if non-KOMA class
  \IfFileExists{parskip.sty}{%
    \usepackage{parskip}
  }{% else
    \setlength{\parindent}{0pt}
    \setlength{\parskip}{6pt plus 2pt minus 1pt}}
}{% if KOMA class
  \KOMAoptions{parskip=half}}
\makeatother
\usepackage{xcolor}
\setlength{\emergencystretch}{3em} % prevent overfull lines
\setcounter{secnumdepth}{5}
% Make \paragraph and \subparagraph free-standing
\ifx\paragraph\undefined\else
  \let\oldparagraph\paragraph
  \renewcommand{\paragraph}[1]{\oldparagraph{#1}\mbox{}}
\fi
\ifx\subparagraph\undefined\else
  \let\oldsubparagraph\subparagraph
  \renewcommand{\subparagraph}[1]{\oldsubparagraph{#1}\mbox{}}
\fi

\usepackage{color}
\usepackage{fancyvrb}
\newcommand{\VerbBar}{|}
\newcommand{\VERB}{\Verb[commandchars=\\\{\}]}
\DefineVerbatimEnvironment{Highlighting}{Verbatim}{commandchars=\\\{\}}
% Add ',fontsize=\small' for more characters per line
\usepackage{framed}
\definecolor{shadecolor}{RGB}{241,243,245}
\newenvironment{Shaded}{\begin{snugshade}}{\end{snugshade}}
\newcommand{\AlertTok}[1]{\textcolor[rgb]{0.68,0.00,0.00}{#1}}
\newcommand{\AnnotationTok}[1]{\textcolor[rgb]{0.37,0.37,0.37}{#1}}
\newcommand{\AttributeTok}[1]{\textcolor[rgb]{0.40,0.45,0.13}{#1}}
\newcommand{\BaseNTok}[1]{\textcolor[rgb]{0.68,0.00,0.00}{#1}}
\newcommand{\BuiltInTok}[1]{\textcolor[rgb]{0.00,0.23,0.31}{#1}}
\newcommand{\CharTok}[1]{\textcolor[rgb]{0.13,0.47,0.30}{#1}}
\newcommand{\CommentTok}[1]{\textcolor[rgb]{0.37,0.37,0.37}{#1}}
\newcommand{\CommentVarTok}[1]{\textcolor[rgb]{0.37,0.37,0.37}{\textit{#1}}}
\newcommand{\ConstantTok}[1]{\textcolor[rgb]{0.56,0.35,0.01}{#1}}
\newcommand{\ControlFlowTok}[1]{\textcolor[rgb]{0.00,0.23,0.31}{#1}}
\newcommand{\DataTypeTok}[1]{\textcolor[rgb]{0.68,0.00,0.00}{#1}}
\newcommand{\DecValTok}[1]{\textcolor[rgb]{0.68,0.00,0.00}{#1}}
\newcommand{\DocumentationTok}[1]{\textcolor[rgb]{0.37,0.37,0.37}{\textit{#1}}}
\newcommand{\ErrorTok}[1]{\textcolor[rgb]{0.68,0.00,0.00}{#1}}
\newcommand{\ExtensionTok}[1]{\textcolor[rgb]{0.00,0.23,0.31}{#1}}
\newcommand{\FloatTok}[1]{\textcolor[rgb]{0.68,0.00,0.00}{#1}}
\newcommand{\FunctionTok}[1]{\textcolor[rgb]{0.28,0.35,0.67}{#1}}
\newcommand{\ImportTok}[1]{\textcolor[rgb]{0.00,0.46,0.62}{#1}}
\newcommand{\InformationTok}[1]{\textcolor[rgb]{0.37,0.37,0.37}{#1}}
\newcommand{\KeywordTok}[1]{\textcolor[rgb]{0.00,0.23,0.31}{#1}}
\newcommand{\NormalTok}[1]{\textcolor[rgb]{0.00,0.23,0.31}{#1}}
\newcommand{\OperatorTok}[1]{\textcolor[rgb]{0.37,0.37,0.37}{#1}}
\newcommand{\OtherTok}[1]{\textcolor[rgb]{0.00,0.23,0.31}{#1}}
\newcommand{\PreprocessorTok}[1]{\textcolor[rgb]{0.68,0.00,0.00}{#1}}
\newcommand{\RegionMarkerTok}[1]{\textcolor[rgb]{0.00,0.23,0.31}{#1}}
\newcommand{\SpecialCharTok}[1]{\textcolor[rgb]{0.37,0.37,0.37}{#1}}
\newcommand{\SpecialStringTok}[1]{\textcolor[rgb]{0.13,0.47,0.30}{#1}}
\newcommand{\StringTok}[1]{\textcolor[rgb]{0.13,0.47,0.30}{#1}}
\newcommand{\VariableTok}[1]{\textcolor[rgb]{0.07,0.07,0.07}{#1}}
\newcommand{\VerbatimStringTok}[1]{\textcolor[rgb]{0.13,0.47,0.30}{#1}}
\newcommand{\WarningTok}[1]{\textcolor[rgb]{0.37,0.37,0.37}{\textit{#1}}}

\providecommand{\tightlist}{%
  \setlength{\itemsep}{0pt}\setlength{\parskip}{0pt}}\usepackage{longtable,booktabs,array}
\usepackage{calc} % for calculating minipage widths
% Correct order of tables after \paragraph or \subparagraph
\usepackage{etoolbox}
\makeatletter
\patchcmd\longtable{\par}{\if@noskipsec\mbox{}\fi\par}{}{}
\makeatother
% Allow footnotes in longtable head/foot
\IfFileExists{footnotehyper.sty}{\usepackage{footnotehyper}}{\usepackage{footnote}}
\makesavenoteenv{longtable}
\usepackage{graphicx}
\makeatletter
\def\maxwidth{\ifdim\Gin@nat@width>\linewidth\linewidth\else\Gin@nat@width\fi}
\def\maxheight{\ifdim\Gin@nat@height>\textheight\textheight\else\Gin@nat@height\fi}
\makeatother
% Scale images if necessary, so that they will not overflow the page
% margins by default, and it is still possible to overwrite the defaults
% using explicit options in \includegraphics[width, height, ...]{}
\setkeys{Gin}{width=\maxwidth,height=\maxheight,keepaspectratio}
% Set default figure placement to htbp
\makeatletter
\def\fps@figure{htbp}
\makeatother

\makeatletter
\makeatother
\makeatletter
\makeatother
\makeatletter
\@ifpackageloaded{caption}{}{\usepackage{caption}}
\AtBeginDocument{%
\ifdefined\contentsname
  \renewcommand*\contentsname{Table of contents}
\else
  \newcommand\contentsname{Table of contents}
\fi
\ifdefined\listfigurename
  \renewcommand*\listfigurename{List of Figures}
\else
  \newcommand\listfigurename{List of Figures}
\fi
\ifdefined\listtablename
  \renewcommand*\listtablename{List of Tables}
\else
  \newcommand\listtablename{List of Tables}
\fi
\ifdefined\figurename
  \renewcommand*\figurename{Figure}
\else
  \newcommand\figurename{Figure}
\fi
\ifdefined\tablename
  \renewcommand*\tablename{Table}
\else
  \newcommand\tablename{Table}
\fi
}
\@ifpackageloaded{float}{}{\usepackage{float}}
\floatstyle{ruled}
\@ifundefined{c@chapter}{\newfloat{codelisting}{h}{lop}}{\newfloat{codelisting}{h}{lop}[chapter]}
\floatname{codelisting}{Listing}
\newcommand*\listoflistings{\listof{codelisting}{List of Listings}}
\makeatother
\makeatletter
\@ifpackageloaded{caption}{}{\usepackage{caption}}
\@ifpackageloaded{subcaption}{}{\usepackage{subcaption}}
\makeatother
\makeatletter
\@ifpackageloaded{tcolorbox}{}{\usepackage[skins,breakable]{tcolorbox}}
\makeatother
\makeatletter
\@ifundefined{shadecolor}{\definecolor{shadecolor}{rgb}{.97, .97, .97}}
\makeatother
\makeatletter
\makeatother
\makeatletter
\makeatother
\journal{Journal Name}
\ifLuaTeX
  \usepackage{selnolig}  % disable illegal ligatures
\fi
\usepackage[]{natbib}
\bibliographystyle{elsarticle-num}
\IfFileExists{bookmark.sty}{\usepackage{bookmark}}{\usepackage{hyperref}}
\IfFileExists{xurl.sty}{\usepackage{xurl}}{} % add URL line breaks if available
\urlstyle{same} % disable monospaced font for URLs
\hypersetup{
  pdftitle={Spatio-temporal dynamics of Microcystis spp. (Cyanophyceae) blooms at Cassino Beach: origin and impact on water quality},
  pdfauthor={Marybeth Costilla; Felipe de Lucia Lobo; João Sarkis Yunes},
  pdfkeywords={AlgaeMAp, cianobactérias
nocivas, ficocianina, pigmentos, sensoriamento remoto},
  colorlinks=true,
  linkcolor={blue},
  filecolor={Maroon},
  citecolor={Blue},
  urlcolor={Blue},
  pdfcreator={LaTeX via pandoc}}

\setlength{\parindent}{6pt}
\begin{document}

\begin{frontmatter}
\title{Spatio-temporal dynamics of \emph{Microcystis spp.}
(Cyanophyceae) blooms at Cassino Beach: origin and impact on water
quality \\\large{Spatio-temporal dynamics of CyanoHAB's} }
\author[1]{Marybeth Costilla%
\corref{cor1}%
}
 \ead{marybeth994@gmail.com} 
\author[2]{Felipe de Lucia Lobo%
%
}
 \ead{felipelobo@ufpel.br} 
\author[1]{João Sarkis Yunes%
%
}
 \ead{joaosarkis@furg.br} 

\affiliation[1]{organization={Universidade Federal do Rio
Grande, Laboratorio de Cianobactérias e Ficotoxinas},addressline={Street
Address},city={Rio Grande},postcode={96203-900},postcodesep={}}
\affiliation[2]{organization={Universidade Federal de
Pelotas, Enghenaria Hídrica},addressline={Street
Address},city={Pelotas},postcode={96110-620},postcodesep={}}

\cortext[cor1]{Corresponding author}



        
\begin{abstract}
\emph{Microcystis} é um gênero de cianobactéria mundialmente conhecida
por produzir hepatoxinas chamadas microcistinas (MC). Os aspectos
nocivos destas florações causam eventos ecológicos e ambientais de
grande impacto. Diversos estudos utilizaram imagens de satélite para
realizar monitoramento da qualidade de água. Assim, imagens de satélite
da Lagoa dos Patos (LP) em condições atmosféricas e hidrológicas
favoráveis, identificam no espaço e tempo as florações de
\emph{Microcystis} que atingem o oceano. Portanto, o objetivo geral
deste projeto é avaliar as respostas bio-óticas das florações de
\emph{Microcystis}; aplicar imagens de satélites para localizar e
determinar a região de origem da espécie que atingem a praia do Cassino,
estimar o conteúdo de toxinas na água e o risco à balneabilidade. Serão
feitas amostragens em pontos estratégicos da saída dos Molhes ao oceano
e na praia do Cassino, para avaliar concentrações de pigmentos, número
de células e identificação taxonômica. Se processarão imagens de
satélite obtidas do sensor OLCI no Sentinel-3, para localizar as
florações e determinar a sua origem no tempo e espaço. Além, se estimará
o risco a balneabilidade da praia do Cassino usando correlações de
concentrações dos pigmentos e número de células. Destacando que estas
florações ocorrem próximos à os cultivos de camarões, mariscos e peixes
(estação de maricultura da FURG), que por bioacumulação de cianotoxinas
representam um risco por consumo a moradores e turistas. Espera-se,
estabelecer uma concentração de clorofila-a para balneabilidade baseado
nos limites estabelecidos na CONAMA 357 e a Organização Mundial da Saúde
(OMS) para águas de recreação.
\end{abstract}





\begin{keyword}
    AlgaeMAp \sep cianobactérias
nocivas \sep ficocianina \sep pigmentos \sep 
    sensoriamento remoto
\end{keyword}
\end{frontmatter}
    \ifdefined\Shaded\renewenvironment{Shaded}{\begin{tcolorbox}[borderline west={3pt}{0pt}{shadecolor}, breakable, sharp corners, interior hidden, boxrule=0pt, enhanced, frame hidden]}{\end{tcolorbox}}\fi

\hypertarget{introduuxe7uxe3o}{%
\section{Introdução}\label{introduuxe7uxe3o}}

Florações de Cianoctérias Nocivas (FCN, ou CyanoHAB's pelo acrônimo em
inglês) causam efeitos ambientais e ecológicos nocivos como redução da
transparência da água, contaminação da água potável, aumento de hipóxia
e anoxia do fundo aquático, perturbação e alteração das redes tróficas
(Qin et al., 2010) \citet{qin}. Além disso, seus metabólitos tóxicos
(cianotoxinas) causam intoxicações graves em aves e mamíferos (inclusive
humanos), afetando os sistemas digestivo, endócrino, dérmico e nervoso
dependendo do tipo de contato ou/e quantidade ingerida (Paerl \& Otten,
2013). As FCN estão relacionadas a distúrbios ecológicos causados pela
urbanização, atividades agrícolas e espécies invasivas (Bykova et al.,
2006). Combinadas com mudanças climáticas, estas florações aumentaram em
intensidade, frequência e distribuição geográfica, nos últimos anos
(Carey et al., 2012; Paerl \& Huisman, 2009; Paerl \& Paul, 2012; Visser
et al., 2016). Resultando em custos econômicos relacionados ao
tratamento da água, turismo, recreação e desvalorização das propriedades
(Hamilton et al., 2014). Com isto, atividades aquáticas recreativas são
realizadas em áreas propensas a florações de cianobactérias. A falta de
opções alternativas leva as pessoas a tolerar condições desencorajadoras
em termos visuais e olfativos (Chorus \& Testai, 2021).

\hypertarget{metodologia}{%
\section{Metodologia}\label{metodologia}}

\hypertarget{uxe1rea-de-estudo}{%
\subsection{Área de Estudo}\label{uxe1rea-de-estudo}}

A Lagoa dos Patos (LP), localizada entre as latitudes 30º12' a 32º12'S,
50º40'a 52º15'W, possui uma extensão de 250 km e largura média de 40 km,
margeando inúmeras cidades e balneários do estado do Rio Grande do Sul
(Möller et al., 2001). Tem influência expressiva na zona costeira
adjacente, aportando à capa superficial (capa de água em contato com o
ar, entre 0 a 1 cm de água), uma variedade de espécies fitoplanctónicas.
A laguna está conectada ao Oceano Atlântico em sua porção sul por um
canal de 700m de largura (Bitencourt et al., 2020). O estuário da LP tem
morfologia do tipo estrangulado e é influenciada por uma micro-maré
(Möller et al., 2001).

\hypertarget{observauxe7uxf5es-e-coletas-amostrais}{%
\subsection{Observações e coletas
amostrais}\label{observauxe7uxf5es-e-coletas-amostrais}}

Entre os próximos meses de verão entre os anos de 2023-2025, as
florações de cianobactérias serão monitoradas na LP através do Programa
AlgaeMAp (Lobo et al.~2021). As florações serão monitoradas in situ em 3
das praias da região de São Lourenço por uma equipe de apoio vinculada
na FURG-Campus São Lourenço, diariamente por observação, baseados em
presencia e ausência em 5 pontos. Uma vez confirmada o alerta da
presença da floração nas praias de SLS, se procederá ao monitoramento
aproximadamente 2 dias após este alerta, percorrendo os 18 km na praia
Cassino, entre os Molhes e o Navio Altair, uma vez que modelos já
construídos para sedimentos finos e partículas flutuantes (que simulam
florações de cianobactérias) indicam que a(o)s mesma(o)s são exportadas
da LP durante períodos de ventos do Norte, sendo acumuladas na praia de
Cassino (Calliari et al., 2008; Canever, 2021). Simultaneamente será
examinado/confirmado esse deslocamento ao Sul, por meio do uso de
imagens de satélites Sentinel-3. As amostragens estão previstas para
ocorrer entre os anos de 2023 a 2025, durante os meses de dezembro a
março (verão) e nos meses de junho a setembro (inverno), considerando
este último período como controle de meses sem floração. No total se
contará com 7 estações de amostragem: 4 estações (a bordo da lancha) nos
molhes e fora deste, na direção ao oceano na boca dos molhes, na saída
do molhes e duas 2 oceanicas e 3 coletas na praia do Cassino com veículo
terrestre (Figura 1).

\begin{figure}

{\centering \includegraphics[width=0.5\textwidth,height=\textheight]{Lagoa_dos_Patos.jpg}

}

\caption{Figura 1.Lagoa dos Patos}

\end{figure}

\hypertarget{bibliography-styles}{%
\section{Bibliography styles}\label{bibliography-styles}}

Here are two sample references: \citet{Feynman1963118} \citet{qin}.

By default, natbib will be used with the \texttt{authoryear} style, set
in \texttt{classoption} variable in YAML. You can sets extra options
with \texttt{natbiboptions} variable in YAML header. Example

\begin{verbatim}
natbiboptions: longnamesfirst,angle,semicolon
\end{verbatim}

There are various more specific bibliography styles available at
\url{https://support.stmdocs.in/wiki/index.php?title=Model-wise_bibliographic_style_files}.
To use one of these, add it in the header using, for example,
\texttt{biblio-style:\ model1-num-names}.

\hypertarget{using-csl}{%
\subsection{Using CSL}\label{using-csl}}

If \texttt{cite-method} is set to \texttt{citeproc} in
\texttt{elsevier\_article()}, then pandoc is used for citations instead
of \texttt{natbib}. In this case, the \texttt{csl} option is used to
format the references. By default, this template will provide an
appropriate style, but alternative \texttt{csl} files are available from
\url{https://www.zotero.org/styles?q=elsevier}. These can be downloaded
and stored locally, or the url can be used as in the example header.

\hypertarget{equations}{%
\section{Equations}\label{equations}}

Para quantificar a presença de cianobactérias em concentrações de
cél/mL, será testado o algoritmo de Cyanobacteria Index (CI),
desenvolvido por Greene (2022), que usando uma sombra espectral (SS) com
uma árvore de decisões, deixa a equação: \[
SS(λ)= ρ_s (λ)-ρ_s (λ^- )+[ρ_s (λ^- )+ ρ_s (λ^+ )]  (λ+λ^-/λ^+ -λ^- )  
\] Onde + e -- indicam uma banda espectral (λs) acima o abaixo do
comprimento de onda usado como alvo (Wynne et al., 2008

\hypertarget{figures-and-tables}{%
\section{Figures and tables}\label{figures-and-tables}}

Figure~\ref{fig-meaningless} is generated using an R chunk.

\begin{figure}

{\centering \includegraphics[width=0.5\textwidth,height=\textheight]{meu_dir_files/figure-pdf/fig-meaningless-1.pdf}

}

\caption{\label{fig-meaningless}A meaningless scatterplot}

\end{figure}

\hypertarget{tables-coming-from-r}{%
\section{Tables coming from R}\label{tables-coming-from-r}}

Tables can also be generated using R chunks, as shown in
Table~\ref{tbl-simple} example.

\begin{Shaded}
\begin{Highlighting}[]
\NormalTok{knitr}\SpecialCharTok{::}\FunctionTok{kable}\NormalTok{(}\FunctionTok{head}\NormalTok{(mtcars)[,}\DecValTok{1}\SpecialCharTok{:}\DecValTok{4}\NormalTok{])}
\end{Highlighting}
\end{Shaded}

\hypertarget{tbl-simple}{}
\begin{longtable}[]{@{}lrrrr@{}}
\caption{\label{tbl-simple}Caption centered above table}\tabularnewline
\toprule\noalign{}
& mpg & cyl & disp & hp \\
\midrule\noalign{}
\endfirsthead
\toprule\noalign{}
& mpg & cyl & disp & hp \\
\midrule\noalign{}
\endhead
\bottomrule\noalign{}
\endlastfoot
Mazda RX4 & 21.0 & 6 & 160 & 110 \\
Mazda RX4 Wag & 21.0 & 6 & 160 & 110 \\
Datsun 710 & 22.8 & 4 & 108 & 93 \\
Hornet 4 Drive & 21.4 & 6 & 258 & 110 \\
Hornet Sportabout & 18.7 & 8 & 360 & 175 \\
Valiant & 18.1 & 6 & 225 & 105 \\
\end{longtable}


\renewcommand\refname{References}
  \bibliography{bibliography.bib}


\end{document}
